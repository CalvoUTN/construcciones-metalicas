\documentclass[../main.tex]{subfiles}

\begin{document}

\section{Cunetas}

\subsection{Condiciones preliminares}
Para dimensionar las cunetas se establecieron las siguientes condiciones:

\begin{equation*}
\text{Para evitar sedimentaciones y/o erosiones} \rightarrow
\begin{cases}
V_{max} = 1,2 \text{m/seg} \\
V_{min} = 0,3 \text{m/seg}
\end{cases}
\end{equation*}

\begin{equation*}
\text{Para asegurar el escurrimiento} \\ \text{sin producir daños en la estructura} \rightarrow
\begin{cases}
H. Agua_{max} = 0,80 \text{m/seg} \\
H. Agua_{min} = 0,25 \text{m/seg}
\end{cases}
\end{equation*}

\subsection{Dimensionamiento}
Para el dimensionamiento se utilizó una cuneta trapezoidal y las expresiones de continuidad y Manning desarrolladas en \cref{continuidad} y \cref{manning}. \cite{presentaciones}

\begin{align}
    V &= \frac{Q}{A} \label{continuidad} \\
    V &= \frac{R^{2/3}*i^{1/2}}{\eta} \label{manning}
\end{align}

Se igualaron las velocidades y se las reordenó de forma tal que en uno de los términos se obtenga lo siguiente:

\begin{equation}
    \frac{Q*\eta}{i^{1/2}} = \frac{[b*h + x*h^2]^{5/3}}{[b + 2*h*(1+x^2)^{1/2}]^{2/3}}
\end{equation}

Luego, por un proceso iterativo se resuelve la ecuación anterior para determinar el valor de $h$, de forma tal que el valor adoptado verifica que la velocidad y el tirante  del agua cumplan con las condiciones iniciales.
\end{document}