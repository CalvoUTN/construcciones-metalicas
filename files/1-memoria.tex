\documentclass[../main.tex]{subfiles}

\begin{document}

\section{Memoria}

Éste trabajo consistió en el diseño y verificación de una ruta en un tramo de aproximadamente 20km. Para esto, dado los parámetros de diseño pre-establecidos y los datos de la nivelación del terreno diseñamos primeramente la traza horizontal del camino, verificando las dos curvas que empalman los tres tramos rectos. \cite{cornero_planimetria}

Luego de ésto, diseñamos la alcantarilla transversal a partir del caudal de diseño obtenido con el Método Racional Generalizado.\cite{metodo_racional} El paso siguiente fue establecer la altimetría del camino de forma tal que el déficit o excedente de suelo sea lo menor posible ($<10\%$), que verificamos por dos métodos: el de áreas modificado y el diagrama de Brukner. \cite{presentaciones} \cite{cornero_altimetria}

Conocido todo lo anterior pudimos últimamente diseñar las curvas verticales que empalman los tramos con distinta pendiente. Esto nos permitió corregir y terminar los $40$ perfiles transversales. 

\end{document}